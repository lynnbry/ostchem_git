\documentclass[12pt]{article}
\usepackage{amsfonts, amsmath, amsthm, amssymb}
\usepackage{fancyhdr, atbegshi, graphicx, subfig, float, versions, verbatim}
\linespread{1.6}
\includeversion{quartic}

\begin{document}
Consider model with positive dilution rate D and positive input nutrient concentration $S^0$. Denote the nutrient, cooperator and cheater by $S, X_1,$ and $X_2$ respectively. The cooperator distributes its energy over three functions: $q_1(e)$ is the fraction of nutrient uptake allocated towards enzyme production which is dependent on the amount of enzyme available, $q_2$ is the fraction of nutrient uptake allocated to ostracism which depends on the quantity of cheaters present, and $q_3$ is the remaining nutrient uptake allocated towards growth. The parameters $q_1(e), q_2(X_2) \in [0,1)$ and $q_1+q_2+q_3=1$, that is, the sum of these functions is necessarily equal to all of the energy available. Thus, we can eliminate $q_3$ from the system, and represent the fraction of energy spent on growth as $1-q_1(e)-q_2(X_2),$ which is assumed to be positive. Ostracism will occur at positive rate  $\Sigma(q_2)$. The per capita uptake rates of cooperator and cheater are assumed to be the same given by $\frac{F(S,E)}{\gamma}$, where $\gamma$ is the yield constant in the conversion of nutrient to new biomass. $\eta_E$ denote the efficiency of the conversion of nutrient to the public good. 

\noindent This leads to the following model: 
\begin{align}
\dot{S}&=D(S^0-S)-\frac{1}{\gamma}F(S,E)(X_1 + X_2)\\
\dot{E}&=\eta_E q_1(E) X_1 F(S,E)-DE\\
\dot{X_1} &= X_1((1-q_1(E)-q_2(X_2))F(S,E)-D)\\
\dot{X_2}&=X_2(F(S,E)-D)- \Sigma (q_2 (X_2)) X_1)
\end{align}

\noindent \textbf{N1:} assume the per capita uptake rate function $F(S,E)$ is non-negative and twice continuously differentiable and satisfies the following assumptions: 
\begin{align*}
&F(0,E)=F(S,0)=0,\\
&F(S,E)>0 \text{ when } S>0 \text{ and } E>0,\\
&\frac{\partial F}{\partial S}(S,E)>0 \text{ and } \frac{\partial F}{\partial E}(S,E)>0 \text { when } S>0 \text{ and } E>0.
\end{align*}

These assumptions imply there is no nutrient uptake when nutrient or public good are absent, that there is nutrient uptake when both are present, and that with increased levels of nutrient or public good there is also an increase in uptake rates. 

\noindent \textbf{N2: }  assume that the cost of ostracism function $q_2(X_2) \in [0,1)$ is non-decreasing, that is, either constant or monotonically increasing. Additionally assume that it satisfies $q_2(0)=0$. Assume further that the rate of ostracism, $\sigma(q_2(x_2))$ is monotonically increasing, that is, more energy spent on ostracism results in an increase in ostracism. Moreover, $\sigma(0)=0$. 

\noindent \textbf{N3: } Further assume that the fraction of nutrient uptake spent on producing enzyme is dependent upon the amount of enzyme present in the environment. That is, $q_1(e)$ will be a decreasing function and $q_1(0)>0$. 

These assumptions imply that there is a cost to ostracize only when cheaters are present and that the ostracism rate is 0 when no cheaters are present. Additionally, that if the environment is saturated with nutrient, the cooperator will produce less of it and when there is a lack of enzyme present the cooperator will produce more of it. 

\noindent We can then scale out the conversion factors $\gamma, \eta_E , \eta_T$ by setting $s=S, e=\frac{E}{\eta_E \gamma}, s^0=S^0, x_i=\frac{x_i}{\gamma}, d=D, f(s,e)=F(s,\eta_E \gamma e)$. 

\noindent We can also set $\sigma(q_2(x_2))= \frac{\Sigma}{X_2}(q_2(x_2))$

This leads to the scaled system: 
\begin{align}
\dot{s}&=d(s^0-s)-(x_1 + x_2) f(s,e) \label{scaleds}\\
\dot{e}&=q_1(e) x_1 f(s,e)-de\\
\dot{x_1} &= x_1((1-q_1(e)-q_2(x_2))f(s,e)-d)\\
\dot{x_2}&=x_2(f(s,e)-d-x_1 \sigma(q_2(x_2)))\label{scaledf}
\end{align}

\noindent Note that $f(s,e) \geq 0 \forall s \geq 0, e \geq 0.$ Additionally, $f(0,e)=f(s,0)=0 \forall s\geq 0, e\geq 0$ and $\frac{\partial f}{\partial s} \geq 0$ and $\frac{\partial f}{\partial e} \geq 0$.  

\noindent This model is well-posed in the following sense that (\ref{scaleds})-(\ref{scaledf}) is dissipative.

\noindent let $m=s+e+x_1+x_2,$, then $$\dot{m}=d(s^0-m)-q_2(x_2 )x_1 f(s,e)- x_1 \sigma(q_2(x_2)) \leq d(s^0-m),$$ and hence $\lim_{\tau \rightarrow +\infty} \sup m(\tau) \leq s^0,$ implying that (\ref{scaleds})-(\ref{scaledf}) is dissipative.

\section{Case 1: Cooperators Only}

\noindent First consider the case where $x_2=0$, that is, cheaters are not present. In this case, there is no ostracism occurring, thus by \textbf{N2}, $q_2(x_2) =0$. Then (\ref{scaleds})-(\ref{scaledf}) is transformed into the system: 
\begin{align}
\dot{s}&=d(s^0-s)-(x_1) f(s,e) \label{nox2s}\\
\dot{e}&=q_1(e) x_1 f(s,e)-de\\
\dot{x_1} &= x_1((1-q_1(e))f(s,e)-d)\label{nox2f}
\end{align}

\noindent Transforming the state of the system from $(s,e,x_1)$ to $(m,e,x_1)$ where $$m=s+e+x_1,$$ we get: 
\begin{align*}
\dot{m}&=d(s^0-m)\\
\dot{e}&=q_1(e) x_1 f(m-e-x_1,e)-de\\
\dot{x_1} &= x_1((1-q_1(e))f(m-e-x_1,e)-d),
\end{align*}

\noindent from which it follows as $\tau \rightarrow +\infty$, $m(\tau ) \rightarrow s^0$. Hence the dynamics are understood by considering the following system: 
\begin{align}
\dot{e}&=q_1(e) x_1 f(s^0-e-x_1,e)-de\label{ex1} \\ 
\dot{x_1} &= x_1((1-q_1(e))f(s^0-e-x_1,e)-d) \label{ex2}
\end{align}

\noindent The steady states, which will be represented of the form $\{e,x_1\}$, can be found by finding solutions to (\ref{ex1}) -(\ref{ex2}). 

\noindent Note that when $x_1 =0$, $\dot{e}=-de -\rightarrow 0$, thus we get the washout stead state $W=\{0,0\}$.

\noindent From (\ref{ex1}) -(\ref{ex2}), we get $$f(s^0-e-x_1,e)=\frac{d}{q_1(e)x_1}e = \frac{d}{1-q_1(e)}.$$ Solving this, we get $$x_1=\frac{1-q_1(e)}{q_1(e)}e.$$ Replacing $x_1$ in our system with this value gives us a function with just one parameter, $e$.

\noindent Let $f(s^0-\frac{1}{q_1(e)}e,e) = h(e),$ and let $\frac{d}{1-q_1(e)} = k(e)$, then solutions to (\ref{ex1}) -(\ref{ex2}) will occur when $h(e)=k(e).$

\noindent Since $\frac{e}{q_1(e)}$ is unbounded, increasing, and 0 when $e=0$, there exists some $e^0$ such that $\frac{e^0}{q_1(e^0)}=s^0.$ Thus, we only consider values of $e$ such that $0<\frac{e}{q_1(e)}<s^0.$

\noindent Notice that \textbf{N1} implies that $h(0)=h(s^0)=0$ and that $h(e)>0$ when $0<e<q_1(e)s^0.$

\noindent When the function $h(e)$ is strictly concave down (i.e. $h''(e)<0$ for $0<e<q_1(e)s^0),$ and the equation $k(e)=\frac{d}{1-q_1(e)}$ is sufficiently small, then there will be two positive solutions. 

\noindent Note that functions of the form $f(s,e)$ can be rewritten as a product of a function of $s$ and $e$ respectively, i.e., $f(s,e)=f_1(s)f_2(e)$. Then $f(s,e)=h(e)=f_1(s^0-\frac{e}{q_1(e)})f_2(e)$. 

\noindent For $h''(e)<0$, it is necessary and sufficient that $g''(e)=\frac{-e}{q_1(e)}<0$. 

\noindent Some examples of functions which satisfy this condition are: 

\noindent 1) $q_1(E)=e^{-E}$ and thus $g(E)=-Ee^{E}$. Here, $g''(E)=-2e^E-Ee^E<0$. 

\noindent 2) $q_1(E)=\frac{1}{1+E^2}$. Then $g(E)=\frac{-E}{q+E^2}$ and $g''(E)=\frac{-2E^3+6E}{(E^2+1)^3}<0$.

\noindent \textbf{N3} Assume $g''(e)<0$ and $k(e)=\frac{d}{1-q_1(e)}$ is sufficiently small. Then it follows, $h''(e)<0$. Thus, there are two positive solutions $e^*$ and $e^{**}$, where $0<e^*<e^{**}<q_1(e)s^0$, such that $h(e)=k(e)$. 

\noindent Thus (\ref{ex1})-(\ref{ex2}) has three steady states. $W=\{0,0,0,0\}, R^*=\{e^*,\frac{1-q_1(e^*)}{q_1(e^*)}e^*\},\\ R^{**}=\{e^{**},\frac{1-q_1(e^{**})}{q_1(e^{**})}e^{**}\}.$

\noindent Next we will determine stability of $R^*$ and $R^{**}$. The linearization of system (\ref{ex1})-(\ref{ex2}) is given by: 
\[ 
J=
\begin{bmatrix}
 x_1 \frac{d q_1(e)}{de} f + x_1 q_1(e) \frac{\partial f}{\partial e} -d  & q_1(e) f  \\
x_1 ((1-q_1(e))\frac{\partial f}{\partial e} - f\frac{d q_1(e)}{d e}) & (1-q_1(e))f -d \\

\end{bmatrix}
\]

where the argument $(s,e)$ is suppressed on $f(s,e)$ to ease notation. 

\noindent Considering the 

\begin{comment}

\noindent 

\pagebreak

\noindent Next, consider the case where $x_2 >0.$

\noindent Let $h(x_2)=\sigma(q_2(x_2))$. 

\noindent Let $\tilde{h}(x_2) = \frac{h(x_2)}{x_2}.$ Then the scaled system can be rewritten as: 

\begin{align}
\dot{s}&=d(s^0-s)-(x_1 + x_2) f(s,e) \label{tildes}\\
\dot{e}&=q_1(e) x_1 f(s,e)-de\\
\dot{x_1} &= x_1((1-q_1(e)-q_2(x_2))f(s,e)-d)\\
\dot{x_2}&=x_2(f(s,e)-d-x_1 \tilde{h}(x_2))\label{tildef}
\end{align}


\noindent Notice that \textbf{N1} implies that $h(0)=h_1((1-q_1)s^0)=0$, and that $h_1(x_1)>0$ when $0<x_1<(1-q_1)s^0$.

\noindent Further assume \textbf{N3: } The function $h_1(x_1)$ is strictly concave down (i.e. $h''(x_1)<0$ for $0<x_1<(1-q_1)s^0$), and the equation $h(x_1)= d/ (1-q_1)$ has exactly two positive solutions $x_1^u$ and $x_1^s$ with $x_1^u<x_1^s$.

\textbf{Lemma 1: } \textit{Assume that \textbf{N1, N2} and \textbf{N3} holds. Then equation (\ref{x1only}) has 3 steady states, $0, x_1^u$ and $x_1^s$. The steady states $0$ and $x_1^s$ are asymptotically stable, and $x_1^u$ is unstable.}

\begin{proof}
The linearization at a steady state of \ref{x1only} is given by: 
\begin{align*}
J=\bigg((1-q_1)[h_1(x_1)+x_1h_1'(x_1)]-d \bigg)
\end{align*}
The determinant of the Jacobian is simply the only entry, so $$\det{(J)}=(1-q_1)[h_1(x_1)+x_1h_1'(x_1)]-d.$$ 
Calculating $\det{(J)}$ at each steady state reveals the stability of each. 

\noindent For $x_1=0$, $\det{(J)}=-d<0$, thus, $x_1=0$ is stable. 

\noindent Note that for $x_1^u$ and $x_1^s$, $h_1(x_1)=d/1-q_1$. Thus $\det{J}=(1-q_1)x_1h'(x_1^u)]$. For $x_1^u$, $h'(x_1^u)>0$, thus, $det{J}>0$ and must have a positive eigenvalue, implying that it is an unstable state. For $x_1^s$, $h'(x_1^s)<0$, and it is thus a stable point.  
\end{proof}

These boundary steady states are also steady states of the full system, thus, for $x_2=0$, we have the steady states of $(s,x_1,x_2)$ as $W=(0,0,0)$, $C_u=(s^u, x_1^u, 0)$ and $C_s=(s^s, x_1^s, 0)$. 



\noindent Now we consider the case where cheaters are present in the full scaled system (\ref{noes})-(\ref{noef}). 

\noindent To find the steady states we set $f(s,\frac{q_1 x_1}{1-q_1-q_2(\sigma )})=d+x_1 \sigma = \frac{d}{1-q_1-q_2(\sigma )}$. This gives us $x_1^*=\frac{d}{\sigma}\frac{q_1+q_2}{1-q_1-q_2}$. 

\noindent Then inserting the found value of $x_1^*$ and $f(s,\frac{q_1 x_1}{1-q_1-q_2(\sigma )})=\frac{d}{1-q_!-q_2}$ into $\dot{s}$ we get: $d(s^0-s)-(x_1^*+x_2)\frac{d}{1-q_1-q_2}=0 \Rightarrow s=s^0-\frac{x_2+x_1^*}{1-q_1-q_2}$. 


\noindent Thus, the steady states of this limiting system can be found by finding the solutions to the following equations: 

\begin{align*}
f(s,e)&=\frac{d}{1-q_1-q_2(\sigma)}\\
x_1&= \frac{d}{\sigma} \big( \frac{q_1+q_2(\sigma )}{1-q_1-q_2(\sigma )}\big) \\
s &= s^0 - \frac{x_2+\frac{d}{\sigma}\big( \frac{q_1+q_2}{1-q_1-q_2(\sigma )} \big)}{1-q_1-q_2(\sigma )}
\end{align*}

\[ 
\tiny
J=
\begin{bmatrix}
-d -(x_1+x_2) \frac{\partial f}{\partial s}& -(x_1+x_2)\frac{\partial f}{\partial e} & -f & -f \\
q_1(e)x_1\frac{\partial f}{\partial s} & x_1 \frac{d q_1(e)}{de} f + x_1 q_1(e) \frac{\partial f}{\partial e} -d + x_1 & q_1(e) f & 0 \\
x_1(1-q_1(e)-q_2(x_2))\frac{\partial f}{\partial s} & x_1 \frac{\partial f}{\partial e}(1-q_2(x_2))-\frac{d q_1(e)}{d e}-q_1(e)) & (1-q_1(e)-q_2(x_2))f -d & -x_1 f \frac{d q_2(x_2)}{d x_2} \\
x_2 \frac{\partial f}{\partial s} & x_2 \frac{\partial f}{\partial e} & 0 & f-d-\sigma(q_2(x_2))-x_2\frac{d \sigma(q_2(x_2))}{dx_2}
\end{bmatrix}
\]

\end{comment}


\end{document}