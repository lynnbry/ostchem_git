\documentclass[12pt]{article}
\usepackage{amsfonts, amsmath, amsthm, amssymb}
\usepackage{fancyhdr, atbegshi, graphicx, subfig, float, versions}
\linespread{1.6}
\includeversion{quartic}

\begin{document}
Consider model with positive dilution rate D and positive input nutrient concentration $S^0$. Denote the nutrient, cooperator and cheater by $S, X_1,$ and $X_2$ respectively. The cooperator ostracizes the cheater at positive rate $\sigma$. Parameters $q_1, q_2(\sigma ) \in [0,1)$ represent the fractions of nutrient uptake that are used for public good production and cost of ostracism. Thus the remaining fraction $(1-q_1-q_2(\sigma ))$ is assumed positive and goes towards growth of the population. The per capita uptake rates of cooperator and cheater are assumed to be the same given by $\frac{F(S,E)}{\gamma}$, where $\gamma$ is the yield constant in the conversion of nutrient to new biomass. $\eta_E$ denote the efficiency of the conversion of nutrient to the public good. 
\noindent This leads to the following model: 
\begin{align}
\dot{S}&=D(S^0-S)-\frac{1}{\gamma}F(S,E)(X_1 + X_2)\\
\dot{E}&=\eta_E q_1 X_1 F(S,E)-DE\\
\dot{X_1} &= X_1((1-q_1-q_2(\sigma ))F(S,E)-D)\\
\dot{X_2}&=X_2(F(S,E)-D- X_1 \sigma)
\end{align}

\noindent \textbf{N1:} assume the per capita uptake rate function $F(S,E)$ is non-negative and twice continuously differentiable and satisfies the following assumptions: 
\begin{align*}
&F(0,E)=F(S,0)=0,\\
&F(S,E)>0 \text{ when } S>0 \text{ and } E>0,\\
&\frac{\partial F}{\partial S}(S,E)>0 \text{ and } \frac{\partial F}{\partial E}(S,E)>0 \text { when } S>0 \text{ and } E>0.
\end{align*}

These assumptions imply there is no nutrient uptake when nutrient or public good are absent, that there is nutrient uptake when both are present, and that with increased levels of nutrient or public good there is also an increase in uptake rates. 

\noindent \textbf{N2: }  assume the ostracism rate is non-negative, $\sigma \geq 0$, and that the cost of ostracism function $q_2(\sigma ) \in [0,1)$ is non-decreasing, that is, either constant or monotonically increasing. Additionally assume that it satisfies $q_2(0)=0$. 

These assumptions imply that there is a cost to ostracize only when then ostracism rate is positive, that is, only when ostracism is occurring. Additionally, that the cost to ostracize cheaters can not decrease with an increases ostracism rate.

\noindent We can then scale out the conversion factors $\gamma, \eta_E , \eta_T$ by setting $s=S, e=\frac{E}{\eta_E \gamma}, s^0=S^0, x_i=\frac{x_i}{\gamma}, d=D, f(s,e)=F(s,\eta_E \gamma e)$. 

This leads to the scaled system: 
\begin{align}
\dot{s}&=d(s^0-s)-(x_1 + x_2) f(s,e) \label{scaleds}\\
\dot{e}&=q_1 x_1 f(s,e)-de\\
\dot{x_1} &= x_1((1-q_1-q_2(\sigma ))f(s,e)-d)\\
\dot{x_2}&=x_2(f(s,e)-d-x_1 \sigma)\label{scaledf}
\end{align}

\noindent Note that $f(s,e) \geq 0 \forall s\geq 0, e\geq 0.$ Additionally, $f(0,e)=f(s,0)=0 \forall s\geq 0, e\geq 0$ and $\frac{\partial f}{\partial s} \geq 0$ and $\frac{\partial f}{\partial e} \geq 0$.  

\noindent This model is well-posed in the following sense that (\ref{scaleds})-(\ref{scaledf}) is dissipative.

\noindent let $m=s+e+x_1+x_2,$, then $$\dot{m}=d(s^0-m)-q_2(\sigma )x_1 f(s,e)- x_1 x_2 \sigma \leq d(s^0-m),$$ and hence $\lim_{\tau \rightarrow +\infty} \sup m(\tau) \leq s^0,$ implying that (\ref{scaleds})-(\ref{scaledf}) is dissipative. 

Transforming the state of the system from $(s,e,x_1,x_2)$ to $(s,z,x_1,x_2)$ where $$z=(1-q_1-q_2(\sigma ))e-q_1x_1,$$ we see that the system is transformed into
\begin{align*}
\dot{s}&=d(s^0-s)-(x_1+x_2)f(s,\frac{z_2+q_1 x_1}{1-q_1-q_2(\sigma )})\\
\dot{z}&=-dz\\
\dot{x_1}&=x_1((1-q_1-q_2(\sigma ))f(s,{\frac{z_2+q_1 x_1}{1-q_1-q_2(\sigma )}})-d)\\
\dot{x_2}&=x_2(f(s,{\frac{z_2+q_1 x_1}{1-q_1-q_2(\sigma )}})-d-x_1 \sigma)
\end{align*}

\noindent as $\tau \rightarrow \infty$, $z(\tau )\rightarrow 0$. Thus we let $z=0$ to obtain the scalar limiting system: 
\begin{align}
\dot{s}&=d(s^0-s)-(x_1+x_2)f(s,\frac{q_1 x_1}{1-q_1-q_2(\sigma )}) \label{noes} \\
\dot{x_1}&=x_1((1-q_1-q_2(\sigma ))f(s,\frac{q_1 x_1}{1-q_1-q_2(\sigma )})-d)\\
\dot{x_2}&=x_2(f(s,\frac{q_1 x_1}{1-q_1-q_2(\sigma )})-d-x_1 \sigma)\label{noef}
\end{align}

\noindent First consider the case where $x_2=0$, that is, cheaters are not present. In this case, there is no ostracism occurring, thus by \textbf{N2}, $\sigma = q_2(\sigma ) =0$. Then (\ref{noes})-(\ref{noef}) is transformed into the system: 
\begin{align}
\dot{s}&=d(s^0-s)-x_1 f(s,\frac{q_1 x_1}{1-q_1})\\
\dot{x_1} &= x_1((1-q_1)f(s,\frac{q_1 x_1}{1-q_1})-d)
\end{align}

\noindent Transforming the state of the system from $(s,x_1)$ to $(m,x_1)$ where $$m=(1-q_1)s+x_1,$$ the system is then transformed into: 
\begin{align}
\dot{m}&=-d(m-(1-q_1)s^0)\\
\dot{x_1} &= x_1((1-q_1)f(\frac{m-x_1}{1-q_1},\frac{q_1 x_1}{1-q_1})-d),
\end{align}

\noindent from which it follows as $\tau \rightarrow +\infty$, $m(\tau ) \rightarrow (1-q_1)s^0$. Hence the dynamics are understood by considering the following system: 
\begin{align*}
\dot{x_1}=x_1((1-q_1)f(s^0- \frac{x_1}{1-q_1},\frac{q_1 x_1}{1-q_1})-d)
\end{align*}
or, written more compactly as 
\begin{align}
\dot{x_1}=x_1((1-q_1)h_1(x_1)-d), \label{x1only}
\end{align}
after defining the function: $$h_1(x_1)=f(s^0- \frac{x_1}{1-q_1},\frac{q_1 x_1}{1-q_1}), 0\leq x_1 \leq (1-q_1)s^0.$$

\noindent Notice that \textbf{N1} implies that $h(0)=h_1((1-q_1)s^0)=0$, and that $h_1(x_1)>0$ when $0<x_1<(1-q_1)s^0$.

\noindent Further assume \textbf{N3: } The function $h_1(x_1)$ is strictly concave down (i.e. $h''(x_1)<0$ for $0<x_1<(1-q_1)s^0$), and the equation $h(x_1)= d/ (1-q_1)$ has exactly two positive solutions $x_1^u$ and $x_1^s$ with $x_1^u<x_1^s$.

\textbf{Lemma 1: } \textit{Assume that \textbf{N1, N2} and \textbf{N3} holds. Then equation (\ref{x1only}) has 3 steady states, $0, x_1^u$ and $x_1^s$. The steady states $0$ and $x_1^s$ are asymptotically stable, and $x_1^u$ is unstable.}

\begin{proof}
The linearization at a steady state of \ref{x1only} is given by: 
\begin{align*}
J=\bigg((1-q_1)[h_1(x_1)+x_1h_1'(x_1)]-d \bigg)
\end{align*}
The determinant of the Jacobian is simply the only entry, so $$\det{(J)}=(1-q_1)[h_1(x_1)+x_1h_1'(x_1)]-d.$$ 
Calculating $\det{(J)}$ at each steady state reveals the stability of each. 

\noindent For $x_1=0$, $\det{(J)}=-d<0$, thus, $x_1=0$ is stable. 

\noindent Note that for $x_1^u$ and $x_1^s$, $h_1(x_1)=d/1-q_1$. Thus $\det{J}=(1-q_1)x_1h'(x_1^u)]$. For $x_1^u$, $h'(x_1^u)>0$, thus, $det{J}>0$ and must have a positive eigenvalue, implying that it is an unstable state. For $x_1^s$, $h'(x_1^s)<0$, and it is thus a stable point.  
\end{proof}

These boundary steady states are also steady states of the full system, thus, for $x_2=0$, we have the steady states of $(s,x_1,x_2)$ as $W=(0,0,0)$, $C_u=(s^u, x_1^u, 0)$ and $C_s=(s^s, x_1^s, 0)$. 

\pagebreak

\noindent Now we consider the case where cheaters are present in the full scaled system (\ref{noes})-(\ref{noef}). 

\noindent To find the steady states we set $f(s,\frac{q_1 x_1}{1-q_1-q_2(\sigma )})=d+x_1 \sigma = \frac{d}{1-q_1-q_2(\sigma )}$. This gives us $x_1^*=\frac{d}{\sigma}\frac{q_1+q_2}{1-q_1-q_2}$. 

\noindent Then inserting the found value of $x_1^*$ and $f(s,\frac{q_1 x_1}{1-q_1-q_2(\sigma )})=\frac{d}{1-q_!-q_2}$ into $\dot{s}$ we get: $d(s^0-s)-(x_1^*+x_2)\frac{d}{1-q_1-q_2}=0 \Rightarrow s=s^0-\frac{x_2+x_1^*}{1-q_1-q_2}$. 


\noindent Thus, the steady states of this limiting system can be found by finding the solutions to the following equations: 

\begin{align*}
f(s,e)&=\frac{d}{1-q_1-q_2(\sigma)}\\
x_1&= \frac{d}{\sigma} \big( \frac{q_1+q_2(\sigma )}{1-q_1-q_2(\sigma )}\big) \\
s &= s^0 - \frac{x_2+\frac{d}{\sigma}\big( \frac{q_1+q_2}{1-q_1-q_2(\sigma )} \big)}{1-q_1-q_2(\sigma )}
\end{align*}





\end{document}